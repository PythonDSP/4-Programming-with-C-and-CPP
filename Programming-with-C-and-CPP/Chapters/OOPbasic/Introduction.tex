\section{Introduction}

Object oriented programming (OOP) increases the re-usability of the code. Also, the codes
become more manageable than non-OOP methods. But, it takes proper planning, and therefore
longer time, to write the codes using OOP method. In this chapter, we will learn following features of OOP along with examples.

\begin{enumerate}
	\item Class and Object
	\item Data encapsulation (or Data hiding)
	\item Inheritance
	\item Polymorphism
	\item Data abstraction
	\item Interface (Abstract class)


\end{enumerate}

\section{Class and object}
A `class' is user defined template which contains variables, constants and functions etc.; whereas an `object' is the instance (or variable) of the class. In simple words, a class contains the structure of the code, whereas the object of the class uses that structure for performing various tasks, as shown in this section.

\section{Create class and object}

\lstinputlisting[
language = C,
caption    = {Create class and object},
label      = {cpp:classObject1}
]{classObject1.cpp}

\lstinputlisting[
language = C,
caption    = {Create class and object},
label      = {cpp:classObject2}
]{classObject2.cpp}

\lstinputlisting[
language = C,
caption    = {Create class and object},
label      = {cpp:classObject3}
]{classObject3.cpp}

\section{Constructor}

\lstinputlisting[
language = C,
caption    = {Constructor},
label      = {cpp:constructorEx}
]{constructorEx.cpp}

\section{Inheritance}

\lstinputlisting[
language = C,
caption    = {Inheritance},
label      = {cpp:InheritanceEx}
]{InheritanceEx.cpp}

\lstinputlisting[
language = C,
caption    = {Inheritance},
label      = {cpp:InheritanceEx2}
]{InheritanceEx2.cpp}

\section{Polymorphism}

\lstinputlisting[
language = C,
caption    = {Polymorphism},
label      = {cpp:PolymorphismEx}
]{PolymorphismEx.cpp}


\section{Abstract class}
\lstinputlisting[
language = C,
caption    = {Abstract class},
label      = {cpp:abstractClassEx}
]{abstractClassEx.cpp}
